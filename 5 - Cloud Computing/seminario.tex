\documentclass{beamer}
\usepackage{graphicx}
\usepackage{caption}
\usepackage{subcaption}
\usepackage{color}
\usepackage{listings}
\usepackage{verbatim}
\captionsetup{compatibility=false}
%\usepackage[italian]{babel}  

%% Se per qualsiasi motivo non riuscissi a compilare col tema Unipd,
%% basta commentare la riga qui sotto e si passa automaticamente al
%% tema beamer di default.
\usetheme{Unipd}

\lstdefinestyle{customc}{
  belowcaptionskip=1\baselineskip,
  breaklines=true,
  %frame=L,
  xleftmargin=\parindent,
  language=C,
  showstringspaces=false,
  basicstyle=\fontsize{7}{7}\selectfont\ttfamily,%\footnotesize\ttfamily,
  keywordstyle=\bfseries\color{green!40!black},
  commentstyle=\itshape\color{gray},
  identifierstyle=\color{blue!40!black},
  %stringstyle=\color{orange},
}
\lstset{escapechar=@,style=customc}

\title{Cloud Computing}
%\subtitle{da inserire}
\author{Diego Casella, Fabio Marcuzzi}
\date{\today}


\begin{document}

\maketitle

\begin{frame}{Contenuti} %Outline
\tableofcontents
\end{frame}

%% Introduzione
\section{Introduzione}
\begin{frame}{Introduzione}
In questo quinto  seminario, analizzeremo quali strumenti e servizi siano disponibili per potere realizzare una
applicazione che sfrutti in pieno le potenzialit\'a offerte dal cloud. Ci concentreremo su di alcuni servizi web pi\'u diffusi ed
alla luce di essi, quali criticit\'a possono insorgere nel software e quali possibili soluzioni adottare.
\end{frame}


%
\section{Amazon Elastic Compute Cloud}
\begin{frame}{Amazon Elastic Compute Cloud}
Amazon Elastic Compute Cloud, d'ora in poi \emph{Amazon EC2} o \emph{EC2} per brevit\'a, \'e un servizio che mette a disposizione dell'utente \emph{capacit\'a di calcolo ridimesionabile}, 
sollevando l'utente dall'onere di investire in hardware performante ed alla sua inevitabile manutenzione.
\end{frame}


\begin{frame}{Amazon EC2 - Caratteristiche}
Le caratteristiche offerte dal servizio \emph{Amazon EC2} sono:
\begin{itemize}
\item ambienti di virtual computing, dette \emph{istanze};
\item template pre-configurati di tali instanze, dette \emph{Amazon Machine Images} o \emph{AMI}, contenenti quanto necessario per avere un server operativo;
\item storage temporaneo, che serve per immagazzinare dati che poi vengono eliminati quando viene spenta o terminata una associata istanza, detto \emph{instance storage volume};
\item storage persistente che utilizza gli \emph{Amazon Elastic Block Store}, o \emph{Amazon EBS}, organizzato in uno o pi\'u \emph{Amazon EBS  volumes};
\end{itemize}
\end{frame}


\begin{frame}{Amazon EC2 - Caratteristiche (2)}
Le caratteristiche offerte dal servizio \emph{Amazon EC2} sono:
\begin{itemize}
\item firewall configurabile;
\item indirizzo IP statico per dynamic cloud computing, denominato \emph{Amazon Elastic IP Address};
\item locazioni fisiche multiple, come ad esempio di istanze o volumi \emph{EBS}, dette regioni o \emph{Availabilty Zones}, che consentono di assegnare istanze a determinate
zone geografiche, ad esempio Europa o Stati Uniti, per migliorare la distribuzione del carico di lavoro e la latenza degli applicativi;
\item metadata, dette anche \emph{tags}, che possono essere create ed assegnate alle varie risorse \emph{Amazon EC2};
\item reti virtuali logicamente separate dal resto del cloud AWS, ma che possono essere connesse alla propria rete, dette \emph{Virtual Private Clouds} o \emph{VPCs};
\end{itemize}
\end{frame}


\section{Amazon Elastic Compute Cloud - API}
\begin{frame}{Amazon Elastic Compute Cloud2 - API}
\begin{exampleblock}{Esempio - AmazonEC2Client}
{\small
AWSCredentials credentials = new AWSCredentials(...);
\newline
AmazonEC2Client client = new AmazonEC2Client( credentials );
\newline
 // allocates an Elastic IP address
\newline
AllocateAddressResult r = client.allocateAddress();
\newline
// copies an existing AMI
CopyImageResult r = client.copyImage(new CopyImageRequest(...));
\newline
// starts an AMI
\newline
client.startInstances(new StartInstancesRequest(...));
\newline
// stops an AMI
\newline
client.stopInstances(new StopInstancesRequest(...));
}
\end{exampleblock}
\end{frame}


\section{Amazon Elastic MapReduce}
\begin{frame}{Amazon Elastic MapReduce}
\emph{Amazon Elastic MapReduce}, abbreviato \emph{Amazon EMR}, \'e un webservice che permette di processare grandi moli di dati. Esso utilizza Apache Hadoop, framework per il calcolo distribuito su server o cluster di servers, il quale si prende carico di assegnare i dati nelle varie istanze di \emph{Amazon EC2}. Viene usato in molteplici ambiti, dal machine learning all'analisi finanziaria.
\end{frame}


\begin{frame}{Amazon Elastic MapReduce}
I vantaggi che \emph{Amazon EMR} fornisce sono:
\begin{itemize}
\item si pu\'o lanciare un cluster \emph{Amazon EMR} in pochi minuti;
\item non serve incaricarsi del setup dei nodi, del/i cluster, di Hadoop, del loro tuning a seconda dei dati processati;
\item \'e scalabile a piacimento, a seconda della dimensione dei dati usati;
\item ogni istanza \'e comunque configurabile singolarmente:  il sistemista ha accesso root in ogni singola macchina;
\item sicurezza, dato che \emph{Amazon EMR} pu\'o essere fatto girare all'interno di una \emph{Amazon VPC}, logicamente separata dall'esterno;
\end{itemize}
\end{frame}


\section{Amazon Elastic MapReduce - API}
\begin{frame}{Amazon Elastic MapReduce - API}
\begin{exampleblock}{Esempio - AmazonElasticMapReduceClient}
{\small
client = AmazonElasticMapReduceClient( new AWSCredentials(...) );
\newline
client.addInstanceGroups(new AddInstanceGroupsRequest(...) );
\newline
client.runJobFlow( new RunJobFlowRequest(...) );
\newline
client.terminateJobFlow( new TerminateJobFlowRequest(...) );
\newline
client.listClusters();
}
\end{exampleblock}
\end{frame}


\section{Possibli Utilizzi dei servizi AWS}
\begin{frame}{Possibli Utilizzi dei servizi AWS}
\'E ragionevole pensare, alla luce di questi servizi visti fino ad ora, di poter evolvere un applicativo pensato per l'utilizzo nel cloud, in maniera tale da potere scalare in maniera
efficiente con la taglia del problema computazionale che gli viene sottoposto. Attraverso l'uso dell'API resa disponibile per i servizi AWS, \'e ipotizzabile poter evolvere un applicativo nel cloud
in maniera tale da poter utilizzare pi\'u potenza di calcolo, quando se ne presenti la necessit\'a, e tutto ci\'o in maniera del tutto trasparente all'utente finale che non sa, e non \'e
necessario che se ne renda conto, di cosa sta avvenendo in background.
\end{frame}


\begin{frame}{Possibli Utilizzi dei servizi AWS}
Sebbene questa evoluzione rappresenti un notevole vantaggio per aumentare le performance dell'applicativo in analisi, ci sono ovviamente altri aspetti da tenere in considerazione, come
il redesign dell'applicativo per poter sfruttare i nuovi servizi, la gestione delle istanze, dei volumi di storage, ecc..
\end{frame}

\section{Q\&A}
\begin{frame}{Q\&A}
\end{frame}

\end{document}
