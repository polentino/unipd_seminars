\documentclass{beamer}
\usepackage{graphicx}
\usepackage{caption}
\usepackage{subcaption}
\usepackage{color}
\usepackage{listings}
\usepackage{verbatim}
\captionsetup{compatibility=false}
%\usepackage[italian]{babel}  

%% Se per qualsiasi motivo non riuscissi a compilare col tema Unipd,
%% basta commentare la riga qui sotto e si passa automaticamente al
%% tema beamer di default.
\usetheme{Unipd}

\lstdefinestyle{customc}{
  belowcaptionskip=1\baselineskip,
  breaklines=true,
  %frame=L,
  xleftmargin=\parindent,
  language=C,
  showstringspaces=false,
  basicstyle=\fontsize{7}{7}\selectfont\ttfamily,%\footnotesize\ttfamily,
  keywordstyle=\bfseries\color{green!40!black},
  commentstyle=\itshape\color{gray},
  identifierstyle=\color{blue!40!black},
  %stringstyle=\color{orange},
}
\lstset{escapechar=@,style=customc}

\title{Iaas/Saas computing, storage}
%\subtitle{da inserire}
\author{Diego Casella, Fabio Marcuzzi}
\date{\today}


\begin{document}

\maketitle

\begin{frame}{Contenuti} %Outline
\tableofcontents
\end{frame}

%% Introduzione
\section{Introduzione}
\begin{frame}{Introduzione}
Quest'oggi parleremo dei \emph{Iaas/SaaS}, e di altri paradigmi \emph{"as a Service"} che sono sorti negli ultimi anni in ambito Web e Cloud.


\end{frame}


% texit o emph non funzionano in section ...
\section{Paradigma ``$as$ $a$ $Service$''}
\begin{frame}{Paradigma ``\emph{as a Service}''}
 Negli ultimi anni sono emersi diversi tipi di paradigmi ``\emph{as a Service}'', differenziatisi per contradistinguere il servizio
 offerto all'utente fruitore di esso. I paradigmi ``\emph{as a Service}'' pi\'u conosciuti sono:
 \begin{itemize}
\item \emph{Software as a Service};
\item \emph{Platform as a Service};
\item \emph{Infrastructure as a Service};
\end{itemize}
\end{frame}

\begin{frame}{Paradigma ``\emph{as a Service}'' - SaaS}
Nel \emph{Software as a Service} l'utente ottiene un SO perfettamente configurato e settato per poter funzionare col
software con il quale intende lavorare. Eventuali update, sia del software che dell'hardware, settaggi e configurazioni vengono
realizzati dal fruitore del servizio.
\begin{exampleblock}{Esempio}
Wordpress fornisce un hosting gratuito per poter realizzare il proprio blog: gli utenti utilizzano l'istanza di Wordpress, senza
doversi preoccupare di effettuare updates dei software correlati (PHP o MYSQL), di migrare i contenuti in webserver pi\'u
capienti ecc...
\end{exampleblock}
\end{frame}


\begin{frame}{Paradigma ``\emph{as a Service}'' - PaaS}
Il \emph{Platform as a Service} fornisce ai clienti un un ambiente predefinto, la ``Platform'',  nella quale essi possono
sviluppare e rilasciare il proprio software in maniera efficiente e veloce.
\begin{exampleblock}{Esempio}
Google App Engine.
\end{exampleblock}
\end{frame}


\begin{frame}{Paradigma ``\emph{as a Service}'' - IaaS}
Nell' \emph{Infrastructure as a Service} gli utenti mettono mano alle macchine stesse - server, storage, networks - fornite sotto
forma di virtual environment, nei quali possono settare gli OS desiderati e internconnetterle come desiderano.
\begin{exampleblock}{Esempio}
Servizio Amazon Web Services.
\end{exampleblock}
\end{frame}


\section{Integrazione di strumenti di terze parti}
\begin{frame}{Integrazione di strumenti di terze parti}
Facendo riferimento allo IaaS, vediamo alcuni strumenti fondamentali per corredare il SaaS di servizi fondamentali e rendere la user experience quanto pi\'u completa ed efficace possibile, in particolare:
\begin{itemize}
\item accounting dell'utilizzo del servizio;
\item gestione dei dati utente (WinSCP va bene, ma si pu\'o fare di meglio);
\end{itemize}
\end{frame}


\subsection{Accounting}
\begin{frame}{Accounting}
Il primo strumento presentato, sebbene sia quello meno visibile all'utente, \'e quello che gestisce l'accounting ed in generale i dati
sensibili di un determinato utente come le credenziali di login, i dati sulla contabilizzazione e la cronologia dell'utilizzo del software.
Tali informazioni devono essere mantenute confidenziali, e tale segretezza va quanto pi\'u possibile mantenuta anche in caso
di falle del sistema da parte di agenti esterni.
\end{frame}

\begin{frame}{Accounting - Windows Azure}
A tal proposito, pensare di salvare tale informazioni direttamente nel profilo utente rappresenta una scelta davvero poco
oculata, per molteplici ragioni:
\begin{itemize}
\item per implementare un sistema di storage \emph{in-house} sicuro e robusto occorre impegnare risorse, intese sia come
sviluppatori, che come tempo;
\item esistono gi\'a servizi online che offrono questo tipo di funzionalit\'a e, essendo specializzate in questo ambito, sono
molto pi\'u sicure ed affidabili di quanto si possa pensare di ottenere con una soluzione \emph{in-house};
\end{itemize}
Per queste serie di motivazioni, \'e ragionevole utilizzare ad esempio il framework \emph{Windows Azure}, che consente di tenere sicure e riservate
le informazioni utente in uno \textit{storage cloud} separato dal server dove vengono eseguite le applicazioni.
\end{frame}


\subsection{Storage}
\begin{frame}{Storage}
Lo storage rappresenta un altro punto focale nell'uso del \textit{software as a service}: l'utente deve essere in grado di inviare
 e ricevere dal proprio computer al server cloud, in
 maniera semplice ed immediata, vari tipi di files: 
\begin{itemize}
\item documenti di testo (es. /\LaTeX), 
\item intere cartelle (progetti firmware, testbench, ecc...)
\item script Python, sorgenti C, ... .
\end{itemize}
\end{frame}

\begin{frame}{Storage - WinSCP}
Strumenti come WinSCP sono utili per trasferire files da un computer locale ad un server remoto ma, essendo tool generici,
mancano di alcune caratteristiche fondamentali per il SaaS, come:
\begin{itemize}
\item la sincronizzazione automatica dei files; 
\item la possibilit\'a di gestirne il versioning.
\end{itemize}
Inoltre, un progetto non viene mai sviluppato da una singola persona ma da un gruppo di lavoro,
 e dunque \'e necessario che gli elementi costitutivi vengano raccolti in un repository comune, al quale tutti abbiano accesso, da
 qualsiasi postazione essi accedano.
\end{frame}

\begin{frame}{Storage - DropBox}
Viste le considerazioni precedenti, un ottimo software che ben si presta allo scopo \'e DropBox. Esso infatti fornisce:
\begin{itemize}
\item servizio di file synchronization;
\item servizio di file versioning;
\item developer API disponibile in Java, utile per integrare le sue funzionalit\'a direttamente nell'applicazione offerta in modalit\'a SaaS.
\end{itemize}
\end{frame}

\begin{frame}{Storage - DropBox API}
\begin{exampleblock}{Esempio - Autenticazione}
{\small
private static final String APP\_KEY = "INSERT\_APP\_KEY";
\newline
private static final String APP\_SECRET = "INSERT\_APP\_SECRET";
\newline
DbxAppInfo appInfo = new DbxAppInfo(APP\_KEY, APP\_SECRET);
\newline
DbxRequestConfig config = new DbxRequestConfig( "com.casella.testapp", "");
\newline
DbxWebAuthNoRedirect webAuth = new DbxWebAuthNoRedirect(config, appInfo);
\newline
String authorizeUrl = webAuth.start();
\newline
// copy the url, get the auth code
\newline
DbxAuthFinish authFinish = webAuth.finish(code);
\newline
DbxClient client = new DbxClient(config, authFinish.accessToken);
}
\end{exampleblock}
\end{frame}

\begin{frame}{Storage - DropBox API}
\begin{block}{DbxClient API}
{\small
DbxEntry	copy(String fromPath, String toPath);
\newline
DbxEntry.Folder	createFolder(String path)
\newline
void	delete(String path)
\newline
DbxEntry.File	getFile(String path, String rev, OutputStream target)
\newline
DbxEntry.WithChildren	getMetadataWithChildren(String path)
\newline
DbxClient.Downloader	startGetFile(String path, String revision)
\newline
DbxClient.Uploader	startUploadFile(String targetPath, DbxWriteMode writeMode, long numBytes)
}
\end{block}

\begin{exampleblock}{Esempio - Visualizzazione File Remoti}
{\small
DbxEntry.WithChildren listing = client.getMetadataWithChildren("/");
\newline
for (DbxEntry child : listing.children) \{
\newline
\hspace*{5 mm}System.out.println( child.name + ": " + child.toString() );
\newline
\}
}
\end{exampleblock}
\end{frame}



\begin{frame}{Storage - EverNote}
Un sottoinsieme particolare del servizio di storage \'e rappresentato dallo storage di appunti o note, che possono essere
sincronizzate e condivise fra pi\'u client. A volte infatti, si pensi per esempio a CfL, condividere file di testo \'e pi\'u che
sufficiente, e l'utilizzo di DropBox in tal caso risulta pi\'u complicato del necessario. Per questa ragione si pu\'o pensare di integrare dei servizi di
notetaking, come ad esempio il famoso EverNote che offre molte funzionalit\'a interessanti fra le quali
\begin{itemize}
\item servizio sincronizzazione delle note;
\item annotazione su PDF;
\item riconoscimento scrittura a mano;
\item developer API disponibile in Java, utile per integrare le sue funzionalit\'a direttamente nell'applicazione offerta in modalit\'a SaaS;
\end{itemize}
\end{frame}

\begin{frame}{Storage - EverNote API}
\begin{exampleblock}{Esempio - Autenticazione}
{\small
private static final String AUTH\_TOKEN = "your developer token";
\newline
EvernoteAuth evernoteAuth = new EvernoteAuth(EvernoteService.SANDBOX,  AUTH\_TOKEN);
\newline
ClientFactory factory = new ClientFactory(evernoteAuth);
\newline
NoteStoreClient noteStore = factory.createNoteStoreClient();
}
\end{exampleblock}
\end{frame}

\begin{frame}{Storage - EverNote API}
\begin{exampleblock}{Esempio - Lettura Note Esistenti}
{\small
List$<$Notebook$>$ notebooks = noteStore.listNotebooks();
\newline
for (Notebook notebook : notebooks) \{
\newline
\hspace*{5 mm}System.out.println("Notebook: " + notebook.getName());
\newline
\hspace*{5 mm}NoteFilter filter = new NoteFilter();
\newline
\hspace*{5 mm}// settaggio filter
\newline
\hspace*{5 mm}...
\newline
\hspace*{5 mm}// cerca le prime 100 note corrispondenti al filtro indicato
\newline
\hspace*{5 mm}NoteList noteList = noteStore.findNotes(filter, 0, 100);
\newline
\hspace*{5 mm}for (Note note : notes) \{
\newline
\hspace*{10 mm} System.out.println( note.getTitle() );
\newline
\hspace*{5 mm} \}
\newline
\}
}
\end{exampleblock}
\end{frame}

\begin{frame}{Storage - Ricapitolazione}
Abbiamo visto finora come sia possibile rendere semplice la sincronizzazione di file, documenti e progetti, che si trovano in pi\'u
locazioni differenti, all'interno dello spazio di lavoro presente nel server cloud. Quello che non abbiamo ancora visto \'e
se sia possibile, ed in tal caso come attuare, una gestione \emph{nel server} del progetto in maniera indipendente da dove esso
sia arrivato, fornendo la possibilt\'a di creare degli \emph{snapshot} dello stato del progetto ogni qual volta l'utente senza
il bisogno di salvare lo stato del progetto in un determinato punto del suo sviluppo.
\end{frame}

\subsection{Versioning}
\begin{frame}{Versioning}
Come conseguenza della considerazione fatta poco fa, si capisce che non \'e importante solo poter inviare, ricevere, sincronizzare
files da pi\'u computer remoti al server cloud; \'e anche necessario fornire un meccanismo di versioning, all'interno del server,
che renda possibile memorizzare il progetto ed i suoi files costituenti, a discrezione dell'utente.
\newline
Per questa ragione si pu\'o pensare di utilizzare Subversion, Git o altri sistemi di version control, e di integrarli nell'applicazione offerta in modalit\'a SaaS, per poter offire una interfaccia semplificata e funzionale all'utente.
\end{frame}

\section{Q\&A}
\begin{frame}{Q\&A}
\end{frame}

\end{document}
